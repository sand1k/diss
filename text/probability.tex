\documentclass{article}

\usepackage{amsmath,amsthm,amssymb}
\usepackage{mathtext}
\usepackage[T1,T2A]{fontenc}
\usepackage[utf8]{inputenc}
\usepackage[english,russian]{babel}
\usepackage{hyperref}


\begin{document}

\section{The empirical mean} 

Define the empirical mean as
$$
\bar X = \frac{1}{n}\sum_{i=1}^n X_i. 
$$
Notice if we subtract the mean from data points, we get data that has mean 0. That is, if we define
$$
\tilde X_i = X_i - \bar X.
$$
The mean of the $\tilde X_i$ is 0.
This process is called "centering" the random variables.
Recall from the previous lecture that the mean is the least squares solution for minimizing
$$
\sum_{i=1}^n (X_i - \mu)^2
$$


\section{The emprical standard deviation and variance}

Define the empirical variance as 
$$
S^2 = \frac{1}{n-1} \sum_{i=1}^n (X_i - \bar X)^2 
= \frac{1}{n-1} \left( \sum_{i=1}^n X_i^2 - n \bar X ^ 2 \right)
$$
The empirical standard deviation is defined as
$s = \sqrt{S^2}$. Notice that the standard deviation has the same units as the data.
The data defined by $X_i / s$ have empirical standard deviation 1. This is called "scaling" the data.

\end{document}
