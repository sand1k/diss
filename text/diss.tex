\documentclass{article}

\usepackage{amsmath,amsthm,amssymb}
\usepackage{mathtext}
\usepackage[T1,T2A]{fontenc}
\usepackage[utf8]{inputenc}
\usepackage[english,russian]{babel}
\usepackage[utf8]{inputenc}
\usepackage{hyperref}
\usepackage{graphicx}
\usepackage{enumitem}

\textheight=24cm           % высота текста
\textwidth=16cm            % ширина текста
\oddsidemargin=0pt         % отступ от левого края
\topmargin=-1.5cm          % отступ от верхнего края
\parindent=24pt            % абзацный отступ
\parskip=0pt               % интервал между абзацами
\tolerance=2000            % терпимость к "жидким" строкам
\flushbottom               % выравнивание высоты страниц
%\def\baselinestretch{1.5} % печать с большим интервалом

\title{Методы статического анализа и оптимизирующих преобразований в компиляторе динамически-типизированного языка}
\author{\copyright Шитов Андрей}
\date{11 января 2018}

\begin{document}

\maketitle
\thispagestyle{empty}
\newpage

\tableofcontents
\newpage

\section{Введение}

JavaScript - основной язык веб-программирования, динамически типизированный язык.
Исторически он зарождался как «язык для склеивания» составляющих частей веб-ресурса: изображений, плагинов, Java-апплетов, который был бы удобен для веб-дизайнеров и программистов, не обладающих высокой квалификацией []. Позднее на нем стали реализовывать все большую часть логики интернет-приложения. Как эффективно выполнять JS? 
Основная техника – специализация типов.
Существующие подходы:
Type feedback (Inline caches) - динамическое профилирование типов
Type inference - статический анализ типов.

Природа динамически-типизированных языков программирования подразумевает, что есть все этапы обработки исходного кода программы (синтаксический анализ, построение промежуточного представления, компиляция) выполняется непосредственно в момент запуска программы. Поэтому статический анализ в полной мере неприменим.

Современные промышленный системы в основном используют динамическое профилирование типов и лишь местами - статический анализ.

%\newpage
%\begin{thebibliography}
%\end{thebibliography}

\end{document}
